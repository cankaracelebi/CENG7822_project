\documentclass[border=10pt]{standalone}
\usepackage{tikz}
\usepackage{amsmath}
\usetikzlibrary{positioning, shapes.geometric, arrows.meta, fit, backgrounds, calc, decorations.pathreplacing}

\begin{document}

\tikzset{
    % Node styles
    img/.style={
        rectangle, 
        draw=gray!70, 
        fill=gray!10, 
        thick,
        minimum width=1.2cm, 
        minimum height=1.8cm,
        rounded corners=3pt,
        font=\small\bfseries
    },
    conv/.style={
        rectangle, 
        draw=yellow!80!black, 
        fill=yellow!25, 
        thick,
        minimum width=0.9cm, 
        minimum height=1.4cm,
        rounded corners=3pt,
        font=\scriptsize\bfseries
    },
    deconv/.style={
        rectangle, 
        draw=blue!70, 
        fill=blue!20, 
        thick,
        minimum width=0.9cm, 
        minimum height=1.4cm,
        rounded corners=3pt,
        font=\scriptsize\bfseries
    },
    fc/.style={
        rectangle, 
        draw=purple!70, 
        fill=purple!15, 
        thick,
        minimum width=0.8cm, 
        minimum height=1cm,
        rounded corners=2pt,
        font=\scriptsize\bfseries
    },
    latent/.style={
        rectangle, 
        draw=green!70!black, 
        fill=green!25, 
        very thick,
        minimum width=1cm, 
        minimum height=1.2cm,
        rounded corners=4pt,
        font=\small\bfseries
    },
    param/.style={
        rectangle, 
        draw=red!70, 
        fill=red!15, 
        thick,
        minimum width=0.7cm, 
        minimum height=0.6cm,
        rounded corners=2pt,
        font=\scriptsize\bfseries
    },
    arrow/.style={
        ->,
        >=Stealth,
        thick,
        draw=gray!60
    },
    label/.style={
        font=\tiny,
        text=gray!70!black
    }
}

\begin{tikzpicture}[node distance=0.5cm and 0.8cm]

    % === ENCODER ===
    % Input image
    \node[img] (input) {};
    \node[below=0.15cm of input, font=\scriptsize] {64$\times$64$\times$3};
    \node[above=0.1cm of input, font=\scriptsize\itshape, text=gray] {Input};
    
    % Conv layers (encoder)
    \node[conv, right=0.8cm of input, minimum height=1.3cm] (conv1) {32};
    \node[below=0.15cm of conv1, font=\tiny, text=gray] {32$\times$32};
    
    \node[conv, right=0.5cm of conv1, minimum height=1.1cm] (conv2) {64};
    \node[below=0.15cm of conv2, font=\tiny, text=gray] {16$\times$16};
    
    \node[conv, right=0.5cm of conv2, minimum height=0.9cm] (conv3) {128};
    \node[below=0.15cm of conv3, font=\tiny, text=gray] {8$\times$8};
    
    \node[conv, right=0.5cm of conv3, minimum height=0.7cm] (conv4) {256};
    \node[below=0.15cm of conv4, font=\tiny, text=gray] {4$\times$4};
    
    % Flatten + FC to mu/sigma
    \node[fc, right=0.8cm of conv4] (fc_enc) {FC};
    \node[below=0.15cm of fc_enc, font=\tiny, text=gray] {4096};
    
    % Mu and Sigma
    \node[param, right=0.7cm of fc_enc, yshift=0.4cm] (mu) {$\boldsymbol{\mu}$};
    \node[param, right=0.7cm of fc_enc, yshift=-0.4cm] (sigma) {$\boldsymbol{\sigma}$};
    \node[below=0.1cm of sigma, font=\tiny, text=gray] {32};
    
    % === LATENT SPACE ===
    \node[latent, right=0.8cm of $(mu)!0.5!(sigma)$] (z) {$\mathbf{z}$};
    \node[below=0.2cm of z, font=\scriptsize, text=green!60!black] {32-dim};
    
    % === DECODER ===
    % FC from z
    \node[fc, right=0.8cm of z] (fc_dec) {FC};
    \node[below=0.15cm of fc_dec, font=\tiny, text=gray] {4096};
    
    % Reshape
    \node[deconv, right=0.5cm of fc_dec, minimum height=0.7cm] (reshape) {256};
    \node[below=0.15cm of reshape, font=\tiny, text=gray] {4$\times$4};
    
    % DeConv layers (decoder)
    \node[deconv, right=0.5cm of reshape, minimum height=0.9cm] (deconv1) {128};
    \node[below=0.15cm of deconv1, font=\tiny, text=gray] {8$\times$8};
    
    \node[deconv, right=0.5cm of deconv1, minimum height=1.1cm] (deconv2) {64};
    \node[below=0.15cm of deconv2, font=\tiny, text=gray] {16$\times$16};
    
    \node[deconv, right=0.5cm of deconv2, minimum height=1.3cm] (deconv3) {32};
    \node[below=0.15cm of deconv3, font=\tiny, text=gray] {32$\times$32};
    
    % Output image
    \node[img, right=0.8cm of deconv3] (output) {};
    \node[below=0.15cm of output, font=\scriptsize] {64$\times$64$\times$3};
    \node[above=0.1cm of output, font=\scriptsize\itshape, text=gray] {Recon.};
    
    % === ARROWS ===
    \draw[arrow] (input) -- (conv1);
    \draw[arrow] (conv1) -- (conv2);
    \draw[arrow] (conv2) -- (conv3);
    \draw[arrow] (conv3) -- (conv4);
    \draw[arrow] (conv4) -- (fc_enc);
    \draw[arrow] (fc_enc.east) -- ++(0.3,0) |- (mu.west);
    \draw[arrow] (fc_enc.east) -- ++(0.3,0) |- (sigma.west);
    \draw[arrow] (mu.east) -- ++(0.3,0) |- (z.west);
    \draw[arrow, dashed] (sigma.east) -- ++(0.3,0) |- (z.west);
    \draw[arrow] (z) -- (fc_dec);
    \draw[arrow] (fc_dec) -- (reshape);
    \draw[arrow] (reshape) -- (deconv1);
    \draw[arrow] (deconv1) -- (deconv2);
    \draw[arrow] (deconv2) -- (deconv3);
    \draw[arrow] (deconv3) -- (output);
    
    % === GROUP LABELS ===
    % Encoder label
    \begin{scope}[on background layer]
        \node[fit=(conv1)(conv2)(conv3)(conv4), draw=yellow!50, fill=yellow!5, rounded corners=8pt, inner sep=6pt] (enc_box) {};
    \end{scope}
    \node[above=0.25cm of enc_box.north, font=\footnotesize\bfseries, text=yellow!70!black] {Encoder (Conv2D)};
    
    % Decoder label
    \begin{scope}[on background layer]
        \node[fit=(reshape)(deconv1)(deconv2)(deconv3), draw=blue!40, fill=blue!5, rounded corners=8pt, inner sep=6pt] (dec_box) {};
    \end{scope}
    \node[above=0.25cm of dec_box.north, font=\footnotesize\bfseries, text=blue!60!black] {Decoder (ConvTranspose2D)};
    
    % Reparameterization trick annotation
    \node[above=0.6cm of z, font=\tiny\itshape, text=green!50!black, align=center] {Reparameterization\\$z = \mu + \sigma \cdot \epsilon$};

\end{tikzpicture}

\end{document}
